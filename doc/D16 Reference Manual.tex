\documentclass{article}
\usepackage{courier}
\usepackage[margin=1in]{geometry}
\title{D16 Processor Reference Manual}
\date{Aug 16, 2016}
\author{Michael Nolan}
\newcommand{\instrr}[2]{
\subsection{#1}
	\begin{tabular}{| c | c | c  | c | c |}
	\hline
	Immediate & opcode & Reserved & source & dest \\ \hline
	Imm & #2 & 00 & rS & rD \\
	\hline
	\end{tabular} \\ \\
	\noindent
	\texttt{#1  rD, \textless rS or immediate\textgreater \\ \\}
}
\newcommand{\instj}[3]{
\subsection{#1}
	\begin{tabular}{| c | c | c | c | c |}
	\hline
	Immediate & opcode & Reserved & condition code & dest \\ \hline
	Imm & #2 & 0 & cc & rD \\
	\hline
	\end{tabular} \\ \\
	\noindent
	\texttt{#1.CC \textless rD#3\textgreater \\ \\}
}
\newcommand{\instr}[3]{
\subsection{#1}
	\begin{tabular}{| c | c | c  | c | c |}
	\hline
	Immediate & opcode & Reserved & source & dest \\ \hline
	Imm & #2 & 00 & 000 & rD \\
	\hline
	\end{tabular} \\ \\
	\noindent
	\texttt{#1 \textless rD#3\textgreater \\ \\} 
}
\newcommand{\inst}[2]{
\subsection{#1}
	\begin{tabular}{| c | c | c |}
	\hline
	Reserved & Opcode & Reserved \\ \hline
	0 & #2 & 00000000 \\ \hline
	\end{tabular}\\ \\ \noindent
	\texttt{#1 \\ \\}
}

\begin{document}
	\pagenumbering{gobble}
	\maketitle
	\newpage
	\tableofcontents
	\newpage
	\pagenumbering{arabic}
	
	\section{The Processor}
	The D16 Processor is a very simple, RISC like 16 bit processor with variable length instructions. It has 8 general purpose registers, 32 special purpose registers, and support for up to 64K of memory.
	\subsection{General Purpose Registers}
	The D16 processor defines 8 general purpose registers, called r0 - r7. 2 of these, although they behave the same as the other registers, have special meaning to the processor and in the ABI, and they are as follows: \newline
		\paragraph{r6:} This is generally used as the pointer to the start of a stack frame, but has no special meaning to the processor
		\paragraph{r7:} This is the stack pointer, and is manipulated via the stack instructions (push and pop)
	\subsection{Flags}
		The processor also contains several flags in Special Register 0. \\ \\
		\begin{tabular}{| l | l |}
		\hline
			Zero & set if the result of the last computation is 0 \\ \hline
			Sign & set if the result is negative (bit 15 is set) \\ \hline
			Carry & set if there was a carry or borrow in the past computation \\ \hline
			oVerflow & set if there was a signed overflow in the last computation \\ \hline
		\end{tabular}
	\section{Instruction Set}
	Most instructions come in 2 formats, register and immediate. The immediate versions of an instruction will have bit 7 set in the opcode field and the 16 bit immediate in the word following the instruction. In the subsequent definitions, op2 will refer to the immediate value if the instruction has an immediate, otherwise it refers to rS.
	\instrr{ADD}{000001}{} 
	rD = rD + op2\\
	Updates flags
	
	\instrr{SUB}{000010}
	rD = rD - op2\\
	Updates flags
	\instr{PUSH}{000011}{ or immediate}\noindent
	r7 = r7 - 2 \\
	memory[r7] = rD \\
	This instruction does not update the flags \\
	\instr{POP}{000100}{}
	rD = memory[r7] \\
	r7 = r7 + 2 \\
	Does not update flags
	\subsection{MOV}
	Mov has 2 different encodings depending whether the immediate (if any) fits into 1 byte. \\
	Neither encoding updates the flags.  
	\subsubsection{general MOV encoding}
	\begin{tabular}{| c | c | c  | c | c |}
	\hline
	Immediate & opcode & Reserved & source & dest \\ \hline
	Imm & 001101 & 00 & rS & rD \\
	\hline
	\end{tabular} \\ \\
	\noindent
	\texttt{MOV rD, \textless rS or immediate\textgreater \\ \\}
	rD = op2 \\
	This encoding is used for register to register MOVs or when the immediate value will not fit in one byte. \\
	\subsubsection{special byte MOV}
	\begin{tabular}{| c | c | c |}
	\hline
	Reserved & opcode & data \\ \hline
	0 & 000101 + rD & byte immediate \\ \hline
	\end{tabular} \\ \\
	\texttt{MOV rD, \textless byte immediate\textgreater \\ \\}
	rD = immediate \\
	This encoding is only used when the immediate will fit in 1 byte
	\instrr{AND}{001110}\noindent
	
	rD = rD AND op2  \\
	This instruction updates the flags, and will reset the overflow and carry flags. \\
	\instrr{OR}{001111}\noindent
	rD = rD OR op2 \\
	This instruction updates the flags, and will reset the overflow and carry flags. \\
	\instrr{XOR}{010000}\noindent
	rD = rD XOR op2 \\
	This instruction updates the flags, and will reset the overflow and carry flags. \\
	\instr{NOT}{010001}\noindent
	rD = !rD (bitwise NOT) \\
	This instruction updates the flags, and will reset the overflow and carry flags. \\
	\instr{NEG}{010010}\noindent
	rD = 0-rD (signed negation) \\
	This instruction updates the flags. \\
	\subsection{LD}
	\begin{tabular}{|c | c | c | c | c | c | c |}
	\hline
	Immediate & opcode & byte & displacement & address & data \\ \hline
	imm & 010011 & byte & disp & rS & rD \\ \hline
	\end{tabular}\noindent \\ \\
	\texttt{LD\{.b\} rD, [rS] \\ LD\{.b\} rD, [immediate] \\ LD\{.b\} rD, [rS+immediate]\\ \\}
	The load instruction loads a word (or byte) of data from the address specified in the brackets into register rD. The byte flag in the instruction encoding is set when the ".b" suffix is present and indicates a byte load. The displacement flag is set when the third instruction form is used and indicates that rS must be added to the displacement when generating the address. The displacement flag should only be set if the immediate flag is also set. If the displacement flag is set and the immediate flag is not, the behavior is undefined. \textbf{Important: All word accesses must be word aligned. Failure to ensure this will result in undefined behavior. }
	\subsection{ST}
	\begin{tabular}{|c | c | c | c | c | c | c |}
	\hline
	Immediate & opcode & byte & displacement & address & data \\ \hline
	imm & 010100 & byte & disp & rS & rD \\ \hline
	\end{tabular}\noindent \\ \\
	\texttt{ST\{.b\} [rS], rD\\ ST\{.b\} [immediate], rD \\ ST\{.b\} [rS+immediate], rD\\ \\}
	The store instruction stores the contents of rD into the address specified inside the brackets. The byte flag in the instruction is set when the ".b" suffix is present, and so the processor will only store the least significant 8 bits into the specified address. Similarly to the LD instruction, the disp flag indicates that the processor should add rS and the following immediate value before using the result as the address. \textbf{Important: All word accesses must be word aligned. Failure to ensure this will result in undefined behavior. }
	\instrr{CMP}{010101}\noindent
	The instruction sets the flags exactly like the SUB instruction, however it does not store the result back to rD.
	\instj{JMP}{010110}{ or immediate}\noindent
	If the condition code evaluates to True, JMP sets the instruction pointer to the address specified. \textbf{Important: This address must be word aligned}
	\instj{CALL}{010111}{ or immediate}\noindent
	If the condition evaluates to true, CALL saves the instruction pointer in the link register (LR), before setting it to the address specified. \textbf{Important: This address must be word aligned}
	\inst{RET}{011000}
	Sets the instruction pointer to the link register(LR).
	\instrr{SHL}{011001}
	Logical left shift rD by op2 and store the result in rD.\\
	Update flags and clear V.
	\instrr{SHR}{011010}
	Logical right shift rD by op2 and store the result in rD.\\
	Update flags and clear V.
	\instrr{ROL}{011011}
	Rotates the bits in rD left by the specified number in op2.\\ Updates flags
	and clears carry and overflow.
	\instrr{RCL}{011100}
	Rotates the bits in rD, plus the carry bit left by the value in op2. \\
	Updates flags and clears overflow.
	\instj{SET}{100001}
	If the condition code evaluates to true, sets rD to 1 otherwise sets rD to 0.
	\instrr{TEST}{100010}
	Sets the flags upon the result of the bitwise and of rD and op2. Does not modify the value of rD.
	\inst{PUSHLR}{100011}
	Pushes the link register onto the stack. This instruction should be used at the beginning of a subroutine if the subroutine executes the \texttt{CALL} instruction anywhere in its body. To return after executing this instruction, execute \\
	\texttt{POP r1\\JMP.AL r1\\}
	\textbf{Note: r1 can be replaced by any GP register}
	\section{Condition Codes}
	In the \texttt{JMP} variety of instructions (\texttt{JMP}, \texttt{CALL}, \texttt{SET}), the condition code field specifies that the instruction should only execute if the expression corresponding to the condition code evaluates to true. The conditions are as follows: \\ \\
	\begin{tabular}{| l | c | l | l |} \hline
	Mnemonic & Encoding & Expression & Meaning \\ \hline
	NV &0000& False & Never execute\\ \hline
	EQ &0001& Z set & Equal\\ \hline
	NE &0010& Z clear & Not equal \\ \hline
	OS &0011& V set & Overflow \\ \hline
	OC &0100& V clear & No overflow \\ \hline
	HI &0101& C set and Z clear & Unsigned greater than \\ \hline
	LS &0110& C clear and Z set & Unsigned less than or equal to \\ \hline
	P  &0111& S clear & Positive \\ \hline
	N  &1000& S set & Negative \\ \hline
	CS &1001& C set & Carry set \\ \hline
	CC &1010& C clear & Carry clear \\ \hline
	GE &1011& S = V & Signed greater than or equal to\\ \hline
	G  &1100& S = V and Z clear & Signed greater than \\ \hline
	LE &1101& Z set or S $\neq$ V & Signed less than or equal to \\ \hline
	L  &1110& S $\neq$ V & Signed less than \\ \hline
	AL &1111& True & Always execute \\ \hline
	\end{tabular}
	\newpage
	\section{Peripherals}
	The D16 processor uses memory mapping to access its peripherals. Currently, the memory mapped region starts at address 0xFF00 and ends at address 0xFFFF. However in subsequent revisions, this may be expanded to the range 0xFC00-0xFFFF, so the programmer is encouraged not to make use of the top 1K of the address space.
	All peripheral registers will specify their size, either 16 bit or 8 bit. \textbf{Important: All registers must be accessed at their stated size. It is illegal to access a 16 bit register as 2 8 bit parts or 2 8 bit registers as a 16 bit register.}
	\subsection{UART}
	The UART is set up as a pair of 8-entry FIFO queues that feed the Tx and Rx circuitry. \\
	\begin{tabular}{|l|l|l| l|}
	\hline
	Name & Size & Address & Description \\ \hline
	\verb|IO_UART_DATA| & 8 &0xFF02 & UART data \\ \hline
	\verb|IO_UART_STATUS| & 8 & 0xFF03 & UART status \\ \hline
	\verb|IO_UART_BAUD| & 16 & 0xFF04 & UART Baud rate divisor \\ \hline
	
	
	\end{tabular}
	\subsubsection{IO\_UART\_DATA}
	\begin{tabular}{|l|l|l|}
	\hline
	Bits & Type & Description \\ \hline
	0-7 & W & Data sent to UART Tx \\ \hline
	0-7 & R & Data from UART Rx \\ \hline
	\end{tabular}
	\subsubsection{IO\_UART\_STATUS}
	\begin{tabular}{|l|l|l|}
	\hline
	Bits & Type & Description \\ \hline
	0 & R & Tx FIFO Space free \\ \hline
	1 & R & Tx FIFO Empty \\ \hline
	2 & R & Rx Data ready \\ \hline
	3 & R & Rx FIFO Overrun \\ \hline
	4-7 & X & Reserved \\ \hline
\end{tabular}
	\subsubsection{IO\_UART\_BAUD}
	\begin{tabular}{|l|l|l|}
	\hline
	Bits & Type & Description \\ \hline
	0-15 & R/W & Baud rate divisor \\ \hline
	\end{tabular}
\end{document}